%!TeX program = xelatex
% CV_Fredrik_Sandhei.tex
%
% (c) Inspired by David J. Grant

\documentclass[letterpaper,11pt]{article}

%-----------------------------------------------------------
%Margin setup

\setlength{\voffset}{0.1in}
\setlength{\paperwidth}{8.5in}
\setlength{\paperheight}{11in}
\setlength{\headheight}{0in}
\setlength{\headsep}{0in}
\setlength{\textheight}{11in}
\setlength{\textheight}{9.5in}
\setlength{\topmargin}{-0.25in}
\setlength{\textwidth}{7in}
\setlength{\topskip}{0in}
\setlength{\oddsidemargin}{-0.25in}
\setlength{\evensidemargin}{-0.25in}
%-----------------------------------------------------------
%\usepackage{fullpage}
\usepackage{xcolor}
\usepackage{calc}
%\textheight=9.0in
\pagestyle{empty}
\raggedbottom
\raggedright
\setlength{\tabcolsep}{0in}

%-----------------------------------------------------------
%Custom commands
\definecolor{lightgray}{gray}{0.8}
\newcommand{\resitem}[1]{\item #1 \vspace{-2pt}}
\newcommand{\resheading}[1]{{\large \noindent\colorbox{lightgray}{\begin{minipage}{\linewidth-2\fboxsep}{\textbf{#1}}\end{minipage}}}}%\parashade[.9]{sharpcorners}{\textbf{#1 \vphantom{p\^{E}}}}}}
\newcommand{\ressubheading}[4]{
\begin{tabular*}{6.5in}{l@{\extracolsep{\fill}}r}
		\textbf{#1} & #2 \\
		\textit{#3} & \textit{#4} \\
\end{tabular*}\vspace{-5pt}}
\newcommand{\resreference}[4]{
\begin{tabular*}{}{}
		\textbf{#1} \\
                #2 \\
		#3 \\
                #4 \\
\end{tabular*}\vspace{-5pt}}
%-----------------------------------------------------------

\begin{document}

\begin{tabular*}{7in}{l@{\extracolsep{\fill}}r}
\textbf{\Large Fredrik Sandhei}  & +47 962 26 718\\
Dyrmyrgata 14B &  fredrik.sandhei@gmail.com \\
3611 Kongsberg 
\end{tabular*}

\vspace{0.1in}

\resheading{Utdanning}
\begin{itemize}
\item
        \ressubheading{Universitetet i Tromsø}{Tromsø}{Bachelor i ingeniørfag: Droneteknologi}{2016 - 2019}
        \begin{itemize}
            \resitem{Uteksaminert med \textbf{A} på bacheloroppgaven.}
            \resitem{Skrev bacheloroppgave om utvikling av et automatisk optisk landingssystem \\for multirotorer.}
        \end{itemize}
\item
	\ressubheading{NTNU}{Trondheim}{Bachelor i fysikk}{Høsten 2015}
\end{itemize}
\resheading{Jobb \& internship}
\begin{itemize}
\item
   \ressubheading{Kongsberg Defence \& Aerospace; Divisjon Missil}{Kongsberg}{Software-ingeniør}{Sep. 2019 - }
	\begin{itemize}
		\resitem{Bidratt på utviklingen av nyere generasjons Naval Strike Missiles.}
		\resitem{Jobber hovedsaklig med utvikling og drift av KONCERTO, et CORBA-\\COMPONENT - rammeverk som alle nye produkter ved DM baseres på. \\KONCERTO leveres på tvers av divisjoner i KDA.}
	\end{itemize}
\item 
	\ressubheading{Norut}{Tromsø}{Internship}{August 2018}
	\begin{itemize}
           \resitem{Bygget tre fixed wing UAV for landmålingsformål m/ Pixhawk (CryoWing Observer - modeller.)}
	\end{itemize}
\item
	\ressubheading{Universitetet i Tromsø}{Tromsø}{Vitenskapelig assistent}{Høsten 2018}
	\begin{itemize}
           \resitem{Øvingslærer i Embedded C for Atmel (AUT - 1001), hvor studentene skulle utvikle på Atmega8 - Mcu.}
           \resitem{Assisterte universitetslektorene ved å holde øvingstimer i AUT - 1001.}
	\end{itemize}

\end{itemize}

\resheading{Skoleprosjekter}
\begin{itemize}
\item
   \ressubheading{Utvikling av Automatic Optical Landing System - 1 (OLS-1)}{Bacheloroppgave}{Universitetet i Tromsø}{Jan. 2019 - Mai 2019}
	\begin{itemize}
           \resitem{Utviklet programvaren for et optisk landingssystem for multirotorer. Programvaren ble skrevet i Python}
           \resitem{Programvaren ble utviklet på en Raspberry Pi og baserte seg på MAVlink - kommunikasjonsprotokollen for open - source autopilotsystemet ArduCopter / ArduPilot.}
	\end{itemize}

\item
	\ressubheading{Design av hjemmeside}{}{Linjeforening for Droneteknologi og automasjon }{}
	\begin{itemize}
           \resitem{Hovedutvikler av nettside for Linjeforeninga for dronetekonologi og automasjon}
           \resitem{Tok initiativet for å danne en nettside for å kunne øke interessen for studiet og danne et nettverk blant medstudentene.}
	\end{itemize}

\item
   \ressubheading{Utvikling og design av fixed wing UAV}{}{Universitetet i Tromsø}{Sept. 2018 - Des. 2018}
	\begin{itemize}
           \resitem{Hadde ansvar for å utvikle og designe en vinge og kropp som skulle oppfylle våre krav til luftdyktighet, ytelse, flygeevne og annet. Benyttet meg av XFLR5 modelleringsapplikasjonen til å beregne løftkoeffisient, teste vingeprofiler med ulike løftkarakteristikker under ulike værforhold.}
           \resitem{Flyet fikk skryt av NORUT for å være laget på en godt strukturert plan med data bak for å støtte opp endelig design.}
	\end{itemize}

\end{itemize}

\resheading{Ferdigheter}

\begin{description}
\item[Programmeringsspråk:]
C/C++ ($\geq$C++11), \LaTeX, Python, shell
\item[Operativsystemer:]
Linux (Manjaro, Red Hat), UNIX, MacOS X, Windows 7 / 10
\item[Applikasjoner:]
Git, ClearCase, \LaTeX, OpenOffice, MS Office, GTest Framework
\item[Diverse:]
software configuration management, agile development, sterk muntlig og skriftlig kommunikasjonsevne, sterk evne til å løse problemstillinger og feilsøke, bra samarbeidsevne innad og utad teamet, TDD, BDD, ADD
\end{description}

\resheading{Interesser}

\begin{description}
\item[Academisk:] Autonom flyvning  programvareutvikling, photonics, microcontrollers, RF/wireless
\item[Sport:] Styrketrening og løping
\item[Software] Jobber foreløpig med utvikling av eget GNU-Make - byggesystem rettet mot utvikling av C++ - applikasjoner. Jobber også med utvikling av egen \textbf{A}* - pathfinder for ArduPilot
\item[Musical:] Spiller gitar og bass
\item[Annet:] Hobby med FPV og modellflyvning. Innehar radiotelefonisertifikatet for luftfart.
\end{description}

\resheading{Referanser}
Leif Richard Skoe \\
Senior Software Engineer \\
Division Missile; Software \\
+ 47 975 42 482 \\
\textbf{leif.richard.skoe@kongsberg.com} \\
-------------------- \\
Knut Petter Svendsen \\
Senior Software Engineer \\
Division Missile; Software \\
+47 905 71 612 \\
\textbf{knut.svendsen@kongsberg.com \\}

\end{document}
